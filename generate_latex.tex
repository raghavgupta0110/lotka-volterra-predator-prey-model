\documentclass{article}

\usepackage{amsmath}
\usepackage{graphics}
\usepackage{graphicx}
\usepackage{hyperref}

\title{Lotka-Volterra Predator / Prey Model}
\author{Raghav Gupta \\
		Rollno. 153079005}

\begin{document}

\input{parameters.tex}

\maketitle
\newpage
\tableofcontents
\newpage
\listoffigures
\newpage

\begin{abstract}
The \clp \init
\end{abstract}

\newpage
\section{Repository}
\section{Dependencies}
\section{Introduction}
The Lotka–Volterra equations, also known as the predator–prey equations, are a pair of first-order, non-linear, differential equations frequently used to describe the dynamics of biological systems in which two species interact, one as a predator and the other as prey. The populations change through time according to the pair of equations\cite{greenwade93}

\url{http://www.sharelatex.com}

\begin{equation*}
\dfrac{dx}{dt} = \alpha x - \beta xy
\end{equation*}
\begin{equation*}
\dfrac{dy}{dt} = \delta x - \gamma xy
\end{equation*}

\begin{figure}[h]
\begin{center}
\includegraphics[scale=0.60]{prey_and_predator.png}
\caption{Prey and Predator Model}
\end{center}
\end{figure}

\begin{figure}[h]
\begin{center}
\includegraphics[scale=0.60]{phase_space_plot.png}
\caption{Phase space plot}
\end{center}
\end{figure}

\newpage
\bibliographystyle{plain}
\bibliography{sample}

\end{document}